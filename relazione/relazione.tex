\documentclass[a4paper]{article}

\makeatletter
\title{Relazione progetto SO\&L}\let\Title\@title
\author{Flavio Ascari}\let\Author\@author
\date{\today}\let\Date\@date

\usepackage[italian]{babel}
\usepackage[utf8]{inputenc}

\usepackage{mathtools}
\usepackage{amssymb}
\usepackage{amsthm}
\usepackage{faktor}
\usepackage{wasysym}

\usepackage[margin=1.5cm]{geometry}
\usepackage{fancyhdr}
\usepackage{subfig}
\usepackage{multirow}

\usepackage{lipsum}
\usepackage{titlesec}
\usepackage{setspace}
\usepackage{mdframed}
\usepackage{aliascnt}

\usepackage{listings}

% Frontespizio e piè di pagina
\pagestyle{fancy}
\fancyhf{}
\rhead{\textsf{\Author}}
\chead{\textbf{\textsf{\Title}}}
\lhead{\textsf{\today}}

% Per avere le sezioni con le lettere
\renewcommand{\thesection}{\Alph{section}}

% indentazione
\setlength{\parindent}{0pt}

% multicols
\usepackage{multicol}
\setlength\columnsep{20pt}
\setlength{\columnseprule}{0,5pt}

% Per l'indice con i link
\usepackage{hyperref}
\hypersetup{linktocpage}


% Formato teoremi, dimostrazioni, definizioni, ... con il ref giusto
\theoremstyle{theorem}
	\newtheorem{theorem}{Teorema}[section]
\theoremstyle{remark}
	\newaliascnt{remark}{theorem}
	\newtheorem{remark}[remark]{Osservazione}
	\aliascntresetthe{remark}
	\providecommand*{\remarkautorefname}{Osservazione}
\theoremstyle{definition}
	\newaliascnt{definition}{theorem}
	\newtheorem{definition}[definition]{Definizione}
	\aliascntresetthe{definition}
	\providecommand*{\definitionautorefname}{Definizione}
\theoremstyle{corollary}
	\newaliascnt{corollary}{theorem}
	\newtheorem{corollary}[corollary]{Corollario}
	\aliascntresetthe{corollary}
	\providecommand*{\corollaryautorefname}{Corollario}
\theoremstyle{lemma}
	\newaliascnt{lemma}{theorem}
	\newtheorem{lemma}[lemma]{Lemma}
	\aliascntresetthe{lemma}
	\providecommand*{\lemmaautorefname}{Lemma}

\newcommand\file[1]{%
	\textbf{#1}}

\newcommand\codeName[1]{%
	\textit{#1}}

\begin{document}

\begin{center}
	\vspace*{0,5 cm}
	{\Huge \textsc{\Title}} \\
	\vspace{0,5 cm}
	\textsc{\Author} \
	\textsc{\Date}
	\thispagestyle{empty}
	\vspace{0,7 cm}
\end{center}
\small

\tableofcontents
\clearpage




\section{Architettura del progetto}
\subsection{Struttura generale}
Inizialmente il \codeName{main} si occupa di leggere il file di configurazione e inizializzare le strutture dati condivise (coda, hashtable, socket, ...). Poi crea i thread che eseguono il server a regime.
La gestione del server avviene con 2 + \codeName{ThreadsInPool} thread:
\begin{itemize}
	\item il \textbf{signal handler} gestisce i segnali. Resta in attesa finché il processo non ne riceve uno, lo gestisce e torna in attesa. Il \codeName{main} diventa questo thread dopo aver creato gli altri. Vedere \autoref{gestione-segnali} per i dettagli.
	\item il \textbf{listener} resta in attesa di nuove connessioni sul socket e/o di ricevere messaggi sulle connessioni già aperte. Se riceve una richiesta, la passa ad un worker tramite la coda \codeName{queue}, così che possa gestire la comunicazione.
	\item i \textbf{worker}, in tutto \codeName{ThreadsInPool}, gesticono le richieste dei client. Restano in attesa sulla coda \codeName{queue}; non appena estraggono qualcosa dalla coda gestiscono la richiesta del client.
\end{itemize}


\subsection{Protocollo di comunicazione client-server}
Su connections.h è documentato il protocollo di comunicazione client-server

Vedere message.h per i dettagli su come vengono mandati i messaggi, le risposte,
i possibili errori, etc...

fd usati:
0, 1, 2: stdin, stdout, stderr
3: il socket su cui accetta connessioni
4: read end della pipe per interrompere la select
5 - 4+MaxConnections: dedicati ai socket su cui si possono connettere i client
5+MaxConnections - 4+MaxConnections+ThreadsInPool: dedicati ai worker per aprire
i file
5+MaxConnections+ThreadsInPool: per scrivere sul file delle statistiche
6+MaxConnections+ThreadsInPool: write end della pipe per interrompere la select

Per evitare problemi con il numero massimo di connessioni (gestite
con i numeri di fd) i fd dei file vengono sempre spostati con dup2 sul valore
MaxConnections + 5 + workerNumber (così si evitano collisioni tra i worker).

\subsection{Comunicazioni interne}
I thread worker e il listener comunicano tramite una coda condivisa su cui
vengono messi i fd dei client che hanno mandato una richiesta. Quando il listener
trova un descrittore pronto, lo inserisce nella coda e lo rimuove dalla bitmap.
Quando un worker finisce di servire un fd lo comunica al listener tramite una
variabile condivisa e gli invia un segnale SIGUSR2 (per interrompere la select).
Quando termina una select, il listener verifica se è stato interrotto da un segnale
(e in questo caso verifica se c'è qualche fd da rimettere nella bitmap) oppure
se c'è qualche nuova connessione in attesa.

Ogni worker e il listener hanno una comunicazione diretta, formata da una
variabile di informazione e un ack. Se l'ack è a 0 vuol dire che il worker può
scrivere nella variabile di informazione, se l'ack è a 1 vuol dire che il
listener deve leggere il valore. Non c'è bisogno di lock su questa comunicazione
perché è a due entità, ognuna che lavora solo se l'ack ha il suo valore e che
una volta finita l'elaborazione cambia l'ack al valore dell'altra (quindi
agiscono per forza alternate).
NON FUNZIONA RACE CONDITION: potrebbe arrivarmi un SIGUSR2 mentre controllo gli
ack a 1 e quindi il listener si blocca sulla select. Quello che vorrei fare è
atomicamente mascherare/smascherare SIGUSR1 e terminare/iniziare la select. La
soluzione è un po' diversa, basata su http://sites.e-advies.nl/unix-signals.html

\subsection{Strutture dati condivise}
Le strutture dati condivise sono:
\begin{itemize}
	\item queue
	\item nickname\_htable
	\item fd\_to\_nickname
	\item 
\end{itemize}
Oltre all'hashtable dei nickname c'è una mappa fd->nickname, utilizzata per
creare l'elenco dei nickname connessi e per gestire le disconnessioni. Questa
mappa è gestita con un array perché UNIX dovrebbe garantire che l'allocazione
dei fd avvenga in ordine crescente, quindi fd\_to\_nickname (la mappa in questione)
è sempre abbastanza piena (e quindi non c'è motivo di usare strutture dati più
complesse). La dimensione di fd\_to\_nickname è MaxConnections + 5: questo,
insieme alla politica di assegnamento dei fd e al fatto che non possono esserci
più di MaxConnections contemporaneamente (il listener le rifiuta), garantisce
che il massimo fd possibile sia MaxConnections + 5 (per stdin, stdout, strerr,
il socket su cui vengono fatte le accept e la pipe). In realtà viene fatto il
contrario: il listener controlla che il nuovo fd sia minore di MaxConnections + 5
e in base a quello stabilisce se le connessioni hanno raggiunto il limite.
Inoltre nel codice tutti i controlli avvengono con MaxConnections (senza +5)
perché dato che viene usato solo aumentato l'operazione viene fatta una volta
per tutte all'avvio del server. Per evitare problemi con le operazioni da file,
i fd su cui vengono aperti i file sono sempre spostati a valori maggiori con
dup2 (questo può causare il rifiuto di un client anche se non si è raggiunto
MaxConnections, ma se il client riprova appena dopo viene accettato).

Per scorrere la history c'è una macro che funziona pensando la history
circolare come replicata dagli indici [0; hist\_size) e [hist\_size; 2 hist\_size),
e scorre da (first + hist\_size) a (first + 1) scendendo. Se la history non
è piena si ferma a hist\_size.


Ci sono tre livelli di lock su chatty per gestire le modifiche dovute ai client:
* la mutex interna dell'hashtable, che serve solo ad evitare che scritture,
rimozioni e distruzione dell'intera hashtable siano sincronizzate (non viene
mai usata esplicitamente, solo tramite la libreria hashtable.h)
* la mutex globale connected\_mutex, che serve per sincronizzare num\_connected,
fd\_to\_nickname e fdnum.
fdnum può essere scritta solo dal listener e, fuori da esso, viene usata solo
come limite superiore per fd\_to\_nickname. Dato che anche questo array può
essere allungato solo dal listener e il listener lo allunga solo per
rispondere ad un nuovo possibile fd (che quindi non è ancora stato passato a
nessun worker) basta assicurarsi di modificare fdnum solo DOPO aver allungato
fd\_to\_nickname per garantire il funzionamento dei worker.
fd\_to\_nickname[fd] viene utilizzato solo in lettura da tutti i worker tranne
quello che sta gestendo fd. Dato che ogni worker gestisce un solo fd, questo
riduce il numero di lock necessarie per accedervi: il worker che sta trattando
fd infatti non ha bisogno di lock in lettura perché nessun altro thread sta
trattando fd, quindi non possono esserci scrittori contemporanei alla sua
lettura. Gli altri thread invece devono sincronizzare le letture per attendere
le scritture dell'eventuale worker che sta gestendo fd.
num\_connected va sincronizzato completamente perché tutti i worker possono
accedervi sia in lettura che in scrittura.
* le mutex dei singoli nickname\_t, che servono a sincronizzare qualsiasi modifica
alla struttura dati.
La garanzia che un solo worker stia gestendo un fd viene dal fatto che il
listener mette in coda solo fd su cui ha aperto nuove connessioni, cosa che può
succedere solo se l'fd è stato chiuso dal worker che lo stava gestendo prima.
Ci sono casi in cui un thread acquisisce sia connected\_mutex che la mutex di un
nickname\_t. Per evitare deadlock, connected\_mutex viene sempre acquisita prima.

\subsection{Gestione dei segnali}\label{gestione-segnali}
Se il gestore dei segnali riceve un segnale SIGUSR1 lo gestisce scrivendo su file
quello che deve scrivere (durante questa operazione non gestisce altri segnali).
Se riceve SIGINT, SIGTERM o SIGQUIT invia un soft break a listener e workers e
ritorna al main, il quale aspetta la fine degli altri threads e poi libera la
memoria allocata.

\subsection{Precisazioni di specifiche}
In caso di errori durante l'esecuzione di una richiesta (es: REGISTER\_OP con
nickname già esistente) il server disconnette il client.

Nonostante quello che fa il client di prova, il server permette UNREGISTER\_OP
solo per deregistrare il nickname con cui si è connessi.

Le statistiche posso farle sceme perché tanto il test è debole: se uno chiede la
sua history non aumenta i messaggi consegnati. Messaggi non consegnati invece
aumenta ogni volta che il tizio a cui devo inviare non è connesso, e non cala
mai. Gli errori aumentano ogni volta che il server risponde picche ad un client:
non crescono se nel server esplode qualcosa ma non viene risposto male al
client, non crescono se il client chiude la connessione di violenza.

\section{Suddivisione del codice}
Il codice è principalmente nel file chatty.c, che contiene tutta l'implementazione delle parti specifiche del server. Avevo pensato di separarlo in almeno due file (chatty.c e worker.c) data la dimensione del codice dei worker, ma l'utilizzo massiccio di variabili globali condivise tra i thread mi ha convinto a lasciare un file unico.

Gli altri file sono stati creati per essere completamente modulari. Ognuno di quei file può essere preso e riutilizzato per un altro progetto (fifo.c l'avevo già scritto per uno degli esercizi svolti durante il corso).

\section{Test}



\section{Varie ed eventuali}
\subsection{Difficoltà incontrate}

\subsection{Cose brutte in giro}
Non c'è nessun tipo di protezione dei file (?!): se qualcuno invia un file con
lo stesso nome di un altro, quello vecchio viene sovrascritto. Ancora meglio, se
due client inviano due file con lo stesso nome insieme il server non si fa
problemi a scrivere sul file da due thread diversi, con tutti gli errori che ne
conseguono. 

\subsection{Test fisici}
Ho provato il programma sul mio computer personale (Debian 10 "Buster", processore x64, 2 core, 4 processori virtuali) e su un altro computer fisso (Debian 9 "Stretch", processore x64, 2 core, 4 processori virtuali). Inoltre sul mio computer personale ho avviato la macchina virtuale del corso (Xubuntu 14.10, )

\subsection{Appunti per i docenti}
Per il test3 \textit{MaxFileSize = 50} in \file{chatty.conf2} con il mio codice non funziona perché \file{client} risulta più grande di 50KiB. Basta cambiare il valore con 70.
\

\

Il test4 fallisce perché cerca di inviare il file \file{listener.o}, che non esiste. Basta cambiare il file con \file{fifo.o} in \file{testleaks.sh}.
\

\

Quando il client stampa la risposta in caso di risposta di errore c'è:
\lstinputlisting[language=C, firstline=212, lastline=212]{../client.c}
ma la condizione è mai vera dato che prima viene eseguita \codeName{readHdr}, quindi \codeName{msg.data} non viene letto.

\end{document}